\chapter{Conclusions}

The primary objective of this thesis was to determine whether genetic algorithms could be utilized to tune hyperparameter values.  In the event that it's feasible, what improvement can be achieved in the models given the computational cost. Research began by reviewing literature and finding similar works published elsewhere. Based on these papers, three machine learning models are selected keeping in mind the computational constraint and the time constraint. Observations conducted by large-scale astronomical surveys record data at an astronomical scale. This data consists of all sorts of astronomical objects and research groups all over the world require data as per their requirements. As a result, the classification of astronomical objects and the creation of detailed catalogues are among the most important and challenging tasks of such observatories.  Every time the data is released, millions of new objects are required to be classified, and machine learning models are well suited for this task because they can be integrated with the data pipeline. The level of automation entails the use of reliable models that are tuned to the specific type of data being handled. 

\section{Model performance}
The baseline performance, i.e., using default hyperparameters of the models of choice, led to very high accuracy scores. This raised the suspicion that the model was overfitting the data and would not be able to generalize, thus resulting in poor classification performance on new data. The study examined concepts such as precision, recall, F1-score, and log loss, and demonstrated the importance of these metrics in quantifying the performance of classification models. The issue of class imbalance emerged as a significant concern, echoing a common real-world scenario in which certain classes have disproportionately fewer instances. In order to mitigate these biases and inaccuracies, stratified sampling was introduced.

In an intriguing discovery, it was found that models using default hyperparameters and genetic algorithms produced similar results. There are several aspects to consider in light of this discovery. In our context, this could very well demonstrate how well-suited the default hyperparameters are to the classification problem by demonstrating their robustness to the particular domain of our dataset. Alternatively, it may indicate that hyperparameter selection and genetic algorithm configurations had no effect on the model's performance. It emphasizes the importance of selecting hyperparameters carefully and fine-tuning them according to domain knowledge.

\section{Ethical considerations}
Despite the fact that my thesis was based on a publicly available astronomical dataset, it is still important to address ethical considerations. While I did not encounter any specific ethical concerns during the course of my research, it is essential to note and discuss the potential ethical implications that may arise in similar studies. The use of a publicly available astronomical dataset may alleviate some ethical concerns, it is still essential to address potential ethical considerations related to data privacy, biases, recognition of contributors, and academic integrity. This thesis utilized a dataset in which no information pertaining to any person, organization, or entity could potentially violate their privacy. Observatories and funding agencies are deeming the SDSS data to be free for public use provided that credit is given where credit is due.

\section{Closing remarks}
This research produced insightful findings and it paves the way for more investigation in the future. It is necessary to investigate increasingly complex hyperparameter spaces and investigate different optimization strategies. With advances in parallel computing technologies, searching complex hyperparameter spaces will become more and more efficient. Future studies are also anticipated to investigate the use of genetic algorithms for a wider variety of optimisation problems than only categorization problems.
This was an exciting journey to study the intersection between machine learning, genetics, and astronomical exploration.